%!TEX root =  ../main.tex

\begin{abstract}
  % 摘要是论文内容的总结概括,应简要说明论文的研究目的、基本研究内容、 研究方法或
  % 过程、结果和结论,突出论文的创新之处。摘要中不宜使用公式、图表,不引用文献。
  % 博士论文中文摘要一般800~1000个汉字,硕士论文中文摘要一般600个汉字。英文摘要的
  % 篇幅参照中文摘要。

  时间(频率)是基本物理量之一,许多物理量与它相关,随着超高精度的冷原子钟以及光钟的发展,
  时钟稳定度已能达到$10^{-18}$量级。如何将如此高精度的时间频率进行远距离传输和比对即为时频
  传递所研究的内容。自由空间中的高精度时频传递在精密测量、基础物理、深空探测、地球物理、
  卫星导航等诸多领域具有非常重要的应用。而相比于传统基于微波技术的时频传递,基于激光的时频传
  递具有更好的性能。另外随着信息社会的发展,时频网络的安全性也得到了更多的关注和研究。
  本论文基于墨子号卫星平台开展了激光时频传输的实验研究,为未来基于量子通信
  网络构建更加安全的时频网络奠定了基础。该研究主要包括以下几个方面:

  1.对异步双向激光时频传递进行了系统性分析,并基于墨子号卫星,首次实验实现了低轨卫星与地面之
  间的异步双向激光测距及时间比对。论文阐述了该方案的基本原理,
  并从激光信号及探测、链路、系统标校等方面进行了误差和稳定度的分析。同时针对其中关键器件我们进行了
  本地测试,例如单光子探测器的时间游走效应测试,时钟性能测试等,并搭建了本地时间比对测试系统。
  利用墨子号卫星现有资源,在星地之间建立了稳定的双向光学链路,获得了8.3kHz的有效异步应答频率,
  通过平滑数据处理,最终实现了1.5cm的测距精度,以及45ps的时间比对精度,验证了该方案在星地
  链路中的可行性。

  2.提出基于卫星共视的激光时频传输方案,该方案有单向链路以及双向链路两种实现方式。单向链路共视
  系统简单,但需要卫星具有高精度定轨能力;而基于异步双向链路共视的方案,能够有效将该项需求消除,
  并实现两地更高精度的时频传递。基于墨子号卫星,我们进行了零基线共视时间比对实验,获得了1.1kHz的
  有效比对频率,实现了25ps的时间比对精度,
  以及5ps@12.6s的比对稳定度,该实验显示了共视时间比对具备良好的系统性能。另外我们还在德令哈
  与丽江之间进行了千公里基线的时间比对演示实验。虽然受限于目前卫星配置,无法在两边实现双向链路,
  但在单向链路的情况下,获得了2.4kHz的有效比对频率,以及15ps的时间比对精度,
  为未来基于卫星共视的激光时频传递提供了一定参考价值。

  3.讨论了基于量子通信网络建立更安全的时频传输网络的可能性。异步双向激光时频传递能够很好地复用
  量子通信卫星的资源并内置于量子通信网络之中,结合利用量子密钥分发提供的安全密钥对时频数据进行
  加密传输,未来能够基于量子通信网络实现一个更加安全的时频网络。

  \keywords{激光;时频传递;测距;双向光学链路;自由空间;安全}
\end{abstract}

\begin{enabstract}
  This is a sample document of USTC thesis \LaTeX{} template for bachelor,
  master and doctor. The template is created by zepinglee and seisman, which
  orignate from the template created by ywg. The template meets the
  equirements of USTC theiss writing standards.

  This document will show the usage of basic commands provided by \LaTeX{} and
  some features provided by the template. For more information, please refer to
  the template document ustcthesis.pdf.

  \enkeywords{University of Science and Technology of China (USTC); Thesis;
  \LaTeX{} Template; Bachelor; Master; PhD}
\end{enabstract}
